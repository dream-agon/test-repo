\documentclass[11pt]{article} % use larger type; default would be 10pt

\usepackage{indentfirst}
\usepackage{abstract}
\usepackage{geometry} 
\usepackage{lettrine} 
\usepackage{multicol}
\usepackage{indentfirst}
\usepackage{cite}
\usepackage{mathtools}
\usepackage{graphicx}
\usepackage{graphics}
\usepackage{subfigure}
\usepackage{caption}
\usepackage{booktabs}
\usepackage{multirow}
\usepackage{diagbox}
\usepackage{makecell}
\usepackage{ctex} % 包含ctex宏包,输入中文所需要
\usepackage{graphicx}%包含图片的宏包
\usepackage{subfigure}%用于排版的宏包
 
\geometry{left=2.0cm,right=2.0cm,top=0.0cm,bottom=2.0cm}
 
\title{系统开发工具基础作业1}
\author{姓名:李俊龙\ \ \ \ \ \ \ \ \ \  学号:23020007061 }
%\date{} % Activate to display a given date or no date (if empty),
         % otherwise the current date is printed 

\begin{document}
\maketitle

%\section{引言}
本报告旨在记录学习git和latex的一些过程以及所遇到的一些问题


%\subsection{git实例1}
(1)首先要了解git是什么:git是一个功能十分强大的分布式版本控制系统,它的最大的特点就在于“版本控制”,可以将一份文档回溯到不同时期的提交状态,想要哪一份文档便可以回溯到那个时期,管理文件的功能十分强大。

%\subsection{git实例2}
(2)用git-bash管理文件:采用'git\ init 文件夹名称'命令可以创造一个由git管理的文件夹,即初始化一个git仓库(该仓库会包含一个.git目录)

%\subsection{git实例3}
(3)将文件纳入版本控制:[1]\ 'git\ add\ *.c',将已经修改完成的文件提交到暂存区。[2]\ git\ commit\ -m“注释”,将暂存区的文件全部提交到远程仓库。
%\subsection{git实例4}

(4)链接远程仓库(以github为例):[1]采用ssh密钥的方式远程链接'github  ssh-keygen -t rsa -C' 这里换上你的邮箱,文件夹中找到'id rsa.pub'文件复制公钥并交给相应的github仓库\\
当然也可以用'git\ remote\ add\ origin 服务器地址'\ 链接某个远程服务器

%\subsection{git实例5}
(5)对提交的文件进行标签化:git log——显示提交记录,并且显示你每次提交所对应的id,这样标签化后方便本地改动

%\subsection[git实例6]
(6)更新与合并:[1]git pull:可以拉取远程仓库的数据,更新本地仓库。
[2]git merge branch
[3]git diff sourcebranch targetbranch:预览分支之间的差异

%\subsection{git课后练习题}
(7)我在github上克隆了我的一个小组作业的仓库,并输入了git stash指令
\begin{figure}[htbp] % 使用figure环境,h表示here
  \centering%使图片居中
    \includegraphics[width=0.5\textwidth]{屏幕截图 2024-08-31 235108 }
\caption{示例}
\end{figure}
并且在输入git log --all --oneline指令后,会显示一个提交列表,从最早到最近。对于快速查看和理解仓库的变更历史非常有用。
  
\begin{figure}[htbp] % 使用figure环境
  \centering%使图片居中
  \includegraphics[width=0.5\textwidth]{屏幕截图 2024-09-01 000451} % 注意路径的斜杠被替换为反斜杠,且路径前加`
  \caption{示例}
\end{figure}

(8)分支命令:分支是用来将特性开发绝缘开来的。在你创建仓库的时候,master 是"默认的"分支。在其他分支上进行开发,完成后再将它们合并到主分支上。\\
比如[1]git branch:没有参数时,git branch 会列出你在本地的分支。\\.[2]git checkout:可以更改分支 \\.[3]git merge:可以合并分支。

接下来是关于学习latex的实例:我选择了一个轻量化的前端TeXworks来编译latex

(1)usepackage{ctex}和 usepackage{graphicx}是需要手动添加的两个宏包,分别含有中文包和图片包

(2)latex中会穿插大量的图片,所以需要学习如何插入图片,一般格式为“begin{figure}[h]h表示herecentering(使图片居中,非必要)\\includegraphics[设置长宽]{图片文件路径}\\caption{示例}”\\end{figure}

(3)Texworks中有很多编译器,最常用的当属XeLaTex,其中有很多模板可以套用

(4)我在插入图片的时候遇到了路径总是出错的情况,在查阅资料之后我选择将我将要上传的图片保存在了与latex文档统一目录下,这样只要输入图片的文件名就可以正确的编译出图片。

(5)图片并排操作如下
\\begin{figure}[htbp]
\\centering
\\subfigure
{
    \\begin{minipage}[b]{.3 linewidth}
        \\centering
        \\includegraphics[scale=0.1]{图片名称}
    \\end{minipage}
}
\\subfigure
{
 	\\begin{minipage}[b]{.3 linewidth}
        \\centering
        \\includegraphics[scale=0.1]{图片名称}
    \\end{minipage}
}
\\subfigure
{
 	\\begin{minipage}[b]{.3 linewidth}
        \\centering
        \\includegraphics[scale=0.1]{图片名称}
    \\end{minipage}
}
\\caption{figure title}
\\end{figure}

                       

\end{document}